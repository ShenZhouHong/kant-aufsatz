% Uncomment for bibliography using biblatex (see includes/formatting.tex)
\section{Bibliography}
All quotations from Immanel Kant's \emph{Critique of Pure Reason} are sourced from the 1996 Werner S. Pluhar translation issued by Hackett Publishing. Citation line numbers refer to the Second (i.e. `B') edition of the \emph{Critique}, as per the \emph{Akademie} manuscript.
% Print every citation in citations.bib, even if unused by \autocite
\nocite{*}
\printbibliography[heading=none]

\section*{Technical Notes}
This essay is typeset using \LaTeX, an Open Source document typesetting language
by Donald Knuth, and version-controlled via Git. The git repository containing notes, source code, and revision history is available at the following link.

% Optional: Include github URL here
\url{https://github.com/ShenZhouHong/kant-aufsatz}

\noindent
This essay is written using the EssayTemplate, an open source \LaTeX\ essay
template designed for the Humanities by Shen Zhou Hong. It is available at:

\url{https://github.com/ShenZhouHong/EssayTemplate}

\noindent
An additional section on the structure of the \emph{Critique} and it's organisation has been omitted from this document for the sake of conciseness. It is available as an optional appendix in the git repository.

\vfill
\begin{center}
This \LaTeX\ essay is also available in Microsoft Word, OpenOffice, HTML, and \mbox{plain text} upon request.
\end{center}
