\section*{The Three Faculties: Sensability, Understanding, Reason}
Before tackling any particular section of Kant's Critique, it is helpful to have an idea of the general structure of Kant's thesis. Kant implicitly divides the human condition\footnote{This is my term, to which I use to refer to the general state of a human being as a thinking, experiential creature.} into three \emph{faculties}, each one having an associated science. These faculties are:

\begin{enumerate}
  \item The faculty of \emph{sensibility}
  \item The faculty of \emph{understanding}
  \item The faculty of \emph{reason}
\end{enumerate}

\noindent
The science of sensibility is \emph{aesthetics}, which is the study of \emph{intuition}, our ability to receive \emph{sensations}. Analogously, the science of understanding is \emph{logic}, which is the study of our ability to generate \emph{concepts}. Finally, in \emph{Transcendental Dialectic}, Kant introduces the faculty of \emph{reason}, which is our means of searching for antecedant causes, (i.e. the \emph{unconditioned}).
% All three faculties have ostentiably mundane (as opposed to \emph{transcendent}, or \emph{transcendental}) forms, which are within the purviews of the physical sciences (e.g. ophthalmology, psychology).
