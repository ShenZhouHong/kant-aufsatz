\section*{The Transcendental Aesthetic}
Our expedition begins with the \emph{transcendental aesthetics} -- that philosophical trailhead, which separates the orderly boulevards of physics from the uncharted wilderness of metaphysics. For what better way there is to explore beyond the physical -- μετὰ τὰ φυσικά -- than by starting with the very foundation of our means of physical perception? Indeed through Kant's introductory division of knowledge as \emph{a priori} and \emph{a posteriori}, we are already driven to aesthetics, as it is the very means in which we know things \emph{a posteriori}.\footnote{A priori knowledge comes \emph{prior} to experience, lit. `from earlier.' A posteriori knowledge comes \emph{posterior} to experience, lit. `from later.'} The faculty of sensibility is shown to us as a means in which we (i.e. the subject, the reasoning human), have the power to intuit \emph{sensations}, which are presented to us by objects that are external to us. The study of sensations, sensibility, and the intuitions which we yield is the science of aesthetics. And it is from this springboard of aesthetics, where Kant begins the Copernican leap that upturns our world.

For we ask ourselves the question: "can there be any a priori knowledge?" To which Kant responds with the demonstration that: although cognitions are composed of concepts which are in reference to objects\footnote{Cognitions arise from the union of understanding with sensibility. Kant explores this in further detail starting at the transcendental analytic.}, all concepts by necessity do not directly refer to their underlying object, but rather only to a sensible \emph{intuition}. However, our faculty of sensibility is never an active one. We never reach out and `grasp' the objects that we wish to sense with some sort of sensory ectoplasm. Rather, the faculty of sensibility is entirely passive -- it is solely our capacity for the \emph{receptivity} of sensation, our ability to be \emph{affected} by objects that are external to us:

\begin{quote}
The capacity (a receptivity) to acquire presentations as a result of the way in which we are affected by objects is called \textbf{sensibility}. Hence by means of sensibility objects are \emph{given} to us, and it alone supplies us with \emph{intuitions}.

\autocite[B33]{hackett}
\end{quote}

\noindent
The objective, material study of our sensibility (i.e. the means in which we are receptive to sensation) is an empirical science, one that is perhaps closer kin to ophthalmology\footnote{The medical science of eyeballs.}, than that of any rank befitting a philosopher. Hence, it is no surprise that Kant dusts away the empirical trappings of a mundane aesthetic science, to ask: "when we abstract away all matter of intuition, what is left there to remain?" For the \emph{matter} of intuition is always an object (and hence, objective, and therefore, empirical) -- to strip intuitions of their matter is to leave only their \emph{form}. There can be only two possibilities regarding the \emph{formal} nature of intuition. Either we are to deny its existence altogether -- as to say that the evaporation of intuitive matter yields only a bare, deposit-less vapour -- or we are to acknowledge that there is indeed some metaphysical residue, a crystallisation of formal structure that underlies all human intuition.

The first case is an apparent impossibility to Kant, one that even a tourist can understand on grounds both metaphysical and mundane. For to argue that there is no formal nature behind human intuition is to accept that all intuitions are inherently structureless. It is to accept that there is no deep, underlying relationship between sensory intuitions -- a nihilistic capitulation so craven that Kant devotes the majority of the Critique's introduction (as well as a good part of his \emph{Prolegomena}) in refuting. He argues that the lack of an a priori, formal structure behind intuition will prevent the apodeictic\footnote{necessarily or self-evidently true.} certainty of mathematics and geometry. Likewise, to use a more mundane analogy -- the mundane, material nature of the human sensory organs by necessity create some formal structure which will underpin our intuition. It is no great leap to take such mundane, aesthetic a prioriae, and to make a further jump that abstracts away even the material nature of our sensory biology -- and ask the \emph{transcendental} question, of what are the a priori forms behind intuition:

\begin{quote}
  There must, therefore, be a science of all principles of a priori sensibility; I call such a science \emph{transcendental aesthetic}. It constitutes the first part of the transcendental doctrine of elements, and stands in contrast to that [part of the] transcendental doctrine of elements which contains the principles of pure thought and is called transcendental logic.

  \autocite[B36]{hackett}
\end{quote}

So what do we find, in this new-found, metaphysical science, of the transcendental aesthetic? What are the underlying a priori forms behind pure intuition? Kant, the ever-obliging tour guide, is happy to answer us, as we begin our hike within the boundaries of the transcendental aesthetic itself. "In the course of that inquiry it will be found that there are two pure forms of sensible intuition, which are principles for a priori cognition: viz., space and time. We now proceed to the task of examining these" \autocite[B37]{hackett}. We are guided to a view of the twin peaks of space (which founds our \emph{outer sense}) and time (which founds our \emph{inner sense}). These mountains serve as the core geology of our transcendental landscape -- we are cautioned against thinking that there are any other -- the lesser distractions of colour, extension, or motion are but superficial molehills which take no part in the a priori bedrock of transcendental aesthetics:

\begin{quote}
  Transcendental aesthetic cannot contain more than these two elements, i.e., space and time. This is evident from the fact that all other concepts belonging to sensibility presuppose something empirical.

  \autocite[B58]{hackett}
\end{quote}

To the untrained glaze of the tourist, it is reasonable for us to ask: "just how firm are the grounds upon which such mountains protrude?" Unable to tell treacherous marsh from wholesome grassland, it is easy for us to assume that Kant's thesis has become an Idealistic one. For surely, should space and time serve as the synthetic a prioriae which underlie all sensibility -- then all of our worldly intuitions are founded upon a mirage -- some dreamy, subjective precept that is real only within our minds. Troubled by such thoughts, we would have been lost in the bogs of Idealism forever, should we not be rescued by Kant's ready alpenstock\footnote{a shepherd’s trekking pole, commonly used to navigate hazardous snowfields and wetlands.}, ever on the watch against Idealism. Space and time are \emph{empirically real}, but \emph{transcendentally ideal}. As the necessary a priori which preconditions intuition, space and time are both universal to all objects of intuition, regardless of subject (i.e. not \emph{subjective}). Likewise, by not being present in the object (i.e. not a property, accident, or attribute) -- but only in the formal nature of our intuition, space and time are \emph{transcendentally ideal}. To use Kant's words in this matter:

\begin{quote}
Hence we assert that space is empirically real (as regards all possible outer experience), despite asserting that space is transcendentally ideal, i.e., that it is nothing as soon as we omit [that space is] the condition of the possibility of all experience and suppose space to be something underlying things in themselves.

\autocite[B45]{hackett}
\end{quote}

Thus concludes the presentation (or to use Euclidean language, the \emph{enunciation}) of the two transcendental elements of space and time, from the metaphysical science of the transcendental aesthetic. The proper, demonstrative proof of these elements are naturally contained within Kant's \emph{Critique} itself, to which this simple tourist's journal dares not to re-attempt. However, naturally it is within the mete of every foreign visitor to question their guide. Hence we will ask (with the characteristic brazenness of a tourist): "to what is the industry of such metaphysical mountains?" Are these mountains mere wastes, suitable for no further purpose -- or do they yield productive quarries, that lead us to further knowledge in metaphysics? These transcendental elements of space and time are ultimately useful, for in their nature as the a priori forms that necessitate all intuition, they define the boundaries for what intuition can yield.

For one, it is readily apparent that should space and time be the necessary condition for intuition, it is impossible for there to be any intuitive knowledge of objects without the forms of space and time. Such knowledge of the \emph{thing-in-itself}, the cognition of an object without the formal conditions of space and time, is impossible. This impossibility of knowing the thing-in-itself serves as a hard boundary for what intuition is capable of, marking all of our intuitive knowledge as knowledge of appearances alone:

\begin{quote}
  [space and time] being merely conditions of sensibility, these a priori sources of cognition determine their own bounds; viz., they determine that they apply to objects merely insofar as these are regarded as appearances, but do not exhibit things in themselves. Appearances are the sole realm where these a priori sources of cognition are valid; if we go outside that realm, there is no further objective use that can be made of them.

  \autocite[B56]{hackett}
\end{quote}

\noindent
And yet, such a border is not a destructive one -- these constraints of the sensible condition are not the concessions of foreign occupiers, partitioning what is rightfully ours. But much like the demarcations of a city's walls, this limit is a productive one -- which allows wealth and industry to flourish. Our empirical world \emph{is} of appearances -- appearances of worldly objects given to the human subject, through the faculty of sensibility. As all of appearance partakes in the shared a prioriae of space and time, there is now a common \emph{synthetic} denominator, universal to all that is which we perceive. Any cognition is always the union of a piece of a posteriori matter, synthetically mated to an a priori form. Now we are able to make broad, general propositions about the empirical objects of sensation, by using the necessary a priori conditions for their appearance as the foundation for our reasoning.

\noindent
It is this very generality, to which Kant takes as the foundation for mathematical intuition. Does it now follow that the universal applicability of mathematics rest in the foundation of these a priori forms behind intuition? Although some may think that the foundation of mathematics is secure now, to which we may close Kant's \emph{Critique} forever -- I caution that the mere existence of a priori forms behind intuition is not enough to sufficiently explain mathematical reasoning. For intuition serves only as the means in which we perceive empirical objects, not as a means in which we understand them. Mathematics deals not just with the intuitive process of appearances (i.e. the appearance of a triangle, that of a derivative), but also a discursive process of seeking the possibility of appearance (i.e. the creation of proofs, application of theorems). Such an act of seeking the possibility of appearance relies not merely on the a priori forms behind possible intuition, but also upon certain acts of understanding, which can only be found in the \emph{transcendental logic}. Hence, the transcendental aesthetic by itself is insufficient to explain the universal applicability of mathematics.

Kant's transcendental treatment of aesthetics is a unique thesis, and indeed a marvellous one. It's metaphysical duality: the synthetic a prioriae both empirically real and transcendentally ideal, is a spectacular marriage that bridges the best parts of Scepticism and Idealism, while avoiding the nihilistic excesses of both. Such a treatment on intuition is already a foundation to mathematical intuition, which is indeed highly spatial and temporal (think of all the figures of geometry, or the integrals of calculus!). Furthermore, as far as intuition serves for the purpose of cognition (where cognition is the union of sensibility and understanding) -- Kant teases us with one final afterthought, which may prove hauntingly prescient in our age of artificial minds:

\begin{quote}
  There is, moreover, no need for us to limit this kind of intuition -- intuition in space and time -- to the sensibility of man. It may be (though we cannot decide this) that any finite thinking being must necessarily agree with man in this regard.

  \autocite[B72]{hackett}
\end{quote}

\noindent
Ultimately, as a work of metaphysical knowledge, the transcendental aesthetics is important to us, because it gives us two important transcendental elements that work with the pure elements of understanding as the basis for cognition. Through this process of analysing the material of sensibility (sensation), we are able to find the a priori concepts of space and time. However, we must not rest on our laurels yet -- for although we have visited this one landmark, there are further mountains ahead, which we must conquer in order to view the entirety of the Kantian island. For intuition alone does not cognition make -- the scion of the union of sensibility and understanding demands equal contribution from both parents. To understand the metaphysical groundwork for cognition, we must next examine the \emph{transcendental logic}. It is only in light of both the \emph{logic} and the \emph{aesthetic}, could we come to a proper assessment of Kant's \emph{Critique}. Our guide hurries us forth -- eager to apply his transcendental method of analysis, to the faculty of understanding. We will not make him wait.

\section*{The Transcendental Logic}
As the faculty of sensibility deals with sensations, the faculty of understanding deals with concepts. Sensibility without understanding is only reaction, while understanding without sensation is but nothing but computation. It is the union of both sensibility and understanding which yields cognition, that fruit of the mind so characteristic of the human animal. As the science of understanding is named logic, Kant brings us to this aptly named summit of the transcendental logic. The journey takes place in two steps. We first acknowledge that the characteristic nature of any concept is that of \emph{function}: "the unity of the act of arranging various presentations under one common presentation" \autocite[B93]{hackett}. Given a raw and unorganised manifold of intuitive presentations, the functional process is one in which such a manifold is unified under a singular concept. That is how a whiff of asphalt, the roar of an engine, and the sparkle of sun-shined chrome all combine to form the singular concept of `race car.'

But yet all such unity is also difference -- for every way in which various presentations are combined, we also perform the negative act of excluding other, unrelated presentations. Only in such a manner do our concepts have any order. This means that concepts are also \emph{judgements} -- acts of discernment and differentiation. Thus lies the basis step of Kant's inductive proof.\footnote{A mathematical analogy. Inductive proofs are completed in two steps, a \emph{basis} step, and an \emph{inductive} step.} As all acts of understanding are concepts, and all concepts judgement\ldots\ therefore all understanding is judgement. "Now since all acts of the understanding can be reduced to judgments, the understanding as such can be presented as a power of judgment" \autocite[B94]{hackett}. This reductive method in which we reduce the problem of finding the a prioriae of concepts to that of judgements is an indispensable one -- as a complete, transcendental analysis of the different modes of judgement is a far easier task, than the analysis of infinite unbounded concepts. We are able to start our analysis by drawing upon the prior scholarship of mundane logic, much like beginning a difficult ascent from a pre-established base camp.

\begin{figure}[H]
  \centering
  \begin{multicols}{2}
    \subsubsection*{Judgement of Quantity}
    \begin{itemize}
      \item Universal
      \item Particular
      \item Singular
    \end{itemize}

    \subsubsection*{Judgement of Quality}
    \begin{itemize}
      \item Affirmative
      \item Negative
      \item Infinite
    \end{itemize}

    \columnbreak
    \subsubsection*{Judgement of Relation}
    \begin{itemize}
      \item Categorical
      \item Hypothetical
      \item Disjunctive
    \end{itemize}

    \subsubsection*{Judgement of Modality}
    \begin{itemize}
      \item Problematic
      \item Assertoric
      \item Apodeictic
    \end{itemize}
  \end{multicols}
  \caption{Kant's Table of Judgements}
  \label{judgements}
\end{figure}


However, upon what grounds do we have to reduce the power of understanding, into a power of judgement? Unlike a mathematical proof that may demonstrate equivalency through formal means (e.g. using the side-angle-side postulate to prove the similarity of two different triangles), this leap from concepts to judgements is an unintuitive\footnote{This is a colloquial usage of the word, no relation to Kantian intuition} one. As tourists unfamiliar to the customs of this land, we are forgiven to voice our suspicions to our guide. There are two problems in Kant's approach. On one hand, the given space of all possible concepts (a concept-space, to borrow a topological vocabulary) may be larger than judgement-space. On the other hand, the four logical judgements that Kant provides may be incomplete. The first question he answers on the basis that the functional nature of concepts makes them formally analogous to judgement (the enunciation which we have walked through in the above paragraph). It is the latter question -- that of whether or not the judgements provided are complete -- which serves as the essential linchpin that holds the Kantian wagon together. Kant himself recognises the dependency of his transcendental logic on furnishing a complete set of judgements:

\begin{quote}
  These concepts [judgements] can be collected in an essay that will be more or less comprehensive \ldots\ but by this -- as it were, mechanical -- procedure, we can never reliably deterine at what point that inquiry will be completed.

  \autocite[B92]{hackett}
\end{quote}

\noindent
Alas, Kant himself never clearly elucidates the reasons why his table of judgements is apodeictically complete. Although it is clear that he borrows the technical apparatus of classical logic, he modifies said apparatus to suit his transcendental use, justifying only his modifications, and not the choice of the apparatus itself. Whether this is because the justification was already apparent to him, or simply an oversight is unknown to us. Kant's judgements hold a certain resemblance to modern fields of mathematical logic. However, there is no single axiomatic formalisation that neatly corresponds to to all four judgements which he lists. The judgements of quality are analogous to propositional calculus, while the judgements of relation are valid forms of logical inference within propositional logic\footnote{We will use this as an analogy in a later section.}. Judgements of quantity are analogous to predicate logic, while the judgements of modality bear the closest resemblance to modal logic. The fact that these Kantian forms of judgement do not correspond neatly to any axiomatically-complete system of logic (for indeed, modal logic as a field has only had axiomatic formalisations in this century) makes it difficult to demonstrate apodeictically that such a table of judgement is complete. Hence, the transcendental analysis of these judgements may not rest upon as certain of a foundation as Kant implies. However, assuming that these judgements are indeed complete, we may move on forth with our journey and not embarrass our guide any further. Let us delve into the pure elements of understanding which we seek.

Using the four logical judgements (or the `logical function in judgements', as per Kant's typical verbosity), we next perform the inductive or transcendental step -- of abstracting away from the content of judgement, to look at it's form. Mired in the difficulty of a rocky ascent, we are forced to take a brief detour to examine the byway of \emph{synthesis}, before the path forward becomes clear. Recall that the characteristic nature of concepts is their unifying function -- that act of uniting presentations. However the very act of judgement (i.e. unifying act) presupposes a priori the existence of a further concept, which gives unity to the synthesis of said disparate presentations. Hence, for all four formal logical judgement, there exists an associated a priori concept which gives synthetic unity to the presentations which underlie the judgement. These transcendental concepts -- now fully abstracted away from any material content of judgement, are the \emph{categories}: the \emph{pure elements of understanding} of the transcendental logic.

\begin{figure}[H]
  \centering
  \begin{multicols}{2}
    \subsubsection*{Category of Quantity}
    \begin{itemize}
      \item Unity
      \item Plurality
      \item Allness
    \end{itemize}

    \subsubsection*{Category of Quality}
    \begin{itemize}
      \item Reality
      \item Negation
      \item Limitation
    \end{itemize}

    \columnbreak
    \subsubsection*{Category of Relation}
    \begin{itemize}
      \item Inherence \& Subsistence
      \item Casuality \& Dependence
      \item Community \& Interaction
    \end{itemize}

    \subsubsection*{Category of Modality}
    \begin{itemize}
      \item Possibility/Impossibility
      \item Existence/Nonexistence
      \item Necessity/Contingency
    \end{itemize}
  \end{multicols}
  \caption{Kant's Table of Categories}
  \label{categories}
\end{figure}


\noindent
These twelve categories which are given to us from the faculty of understanding serve as the same transcendental foundation for conception, as the a priori forms of space and time relate to intuition. Having met these elements which compose the second part of the transcendental doctrine of elements, we are now once again left to ask of their utility. "What do these categories tell us about metaphysics? Are space, time, and the categories sufficient grounds for a complete metaphysical foundation?" The answer to this question is a nuanced one. For the transcendental elements of space, time, and category, are still insufficient in isolation. As with chemical elements, they are inert in their pure, transcendentally isolated state. They are the mere feedstock of reactions, which yield the dizzying chemistry that is cognition. For as we have hinted at throughout our journey, it is the union of intuition and concept which leads to cognition.

This union is conceived of originally as a mundane\footnote{As opposed to transcendental.} one. For it seems that the imposition of a sensible intuition upon understanding leads by it's very nature to an \emph{empirical concept}. That is, a cognition which refers to an object of intuition. This would make this marriage a metaphysically barren one, for such empirical cognitions are fundamentally \emph{objective}, and hence contingent -- there is no space for a priori knowledge here. There can be no morganatic marriage\footnote{A marriage between persons of unequal social rank, e.g. Archduke Franz Ferdinand and Sophie Chotek.} in which we can salvage the union -- for the nature of a category is to bring differing contents into synthesis, and the pure element of intuition are by their definition, contentless. Our exploration of the transcendental landscape would end by necessity, at this insurmountable cliff. This appears to be the case, at least to the untrained eyes of a visitor. However, Kant is able to solve this seemingly impossible unity, through an act of transcendental clarity.

\noindent
Recall that empirical concepts do not refer to singular, individual intuitions -- but rather, they refer to manifolds of intuition, as exemplified by the earlier `race car' example. These manifolds of intuition, which we talk about in the transcendental sense (i.e abstracted away from their content) still contain the a priori forms of space and time, as these are the necessary conditions to intuition. The existence of these a priori manifolds, which are not experience, but only the possibility of experience\footnote{Kant sometimes uses the word presentation, which is a more general form that also encompasses non-intuitive sources.} (as they contain the formal a prioriae of space and time) -- yields a transcendental manifold which is capable of union with the categories. But what manner of a manifold is this? An abstract and unconcerned manifold\footnote{A manifold that does not contain an experiential subject. I have not explained this distinction yet, but it will become clear in due course.} of possible experience cannot ever be united in synthesis with the categories of pure understanding. Categories are just the a priori forms of judgements, e.g. the form of $((P \lor Q) \land \neg P) \to Q$\footnote{propositional notation for a disjunctive judgement}. Even though the intuitive manifold provides us with the propositional terms $P, Q$ (to extend the propositional logic metaphor), and the categories provide us with all the brackets and squiggly signs; the ultimate logical implication (denoted by the symbol `$\to$') is found in neither. Hence, with a simply unconcerned manifold, there is can be no union, and hence no cognition.

This leads us to what is perhaps \emph{the} most thought-provoking idea in Kant's entire \emph{Critique of Pure Reason}, which is the \emph{synthetic unity of apperception}. This synthetic unity of apperception is the unifying act which is capable of bringing intuition together with concepts, yielding cognition. It is the `$\to$' symbol of every judgement. "This is what the little relational word \emph{is} in judgements intends [to indicate], in order to distinguish the objective unity of given presentations [the manifold] from the subjective one" \autocite[B142]{hackett} Kant leads us on, explaining:

\begin{quote}
  The \emph{I think} must be \emph{capable} of accompanying all my presentations. For otherwise something would be presented to me that could not be thought at all \ldots\ presentation that can be given prior to all thought is called \emph{intuition}.

  Hence everything manifold in intuition has a necessary reference to the \emph{I think} in the same subject in whom this manifold is found.\footnote{This subject is the thinking self, i.e. the human being.} But this presentation [i.e., the \emph{I think}] is an act of spontaneity; i.e., it cannot be regarded as belonging to sensibility. I call it \emph{pure apperception}.

  \autocite[B132]{hackett}
\end{quote}

\noindent
Kant goes on to further elaborate regarding the offspring of such a union, which draw their resemblance from space (corresponding to the outer sense) and time (corresponding to the inner sense), in sections entitled \emph{On the Synthesis of Apprehension in Intuition}, and the \emph{On the Synthesis of Reproduction in Imagination}. However these discussions are merely the procedural lemmae of his transcendental analytic, contingent on the synthetic unity of apperception, and we will not elaborate upon them here. Instead we will spend the rest of our time examining the synthetic unity of apperception, this unifying force which marries sensibility and understanding.

\section*{The Synthetic Unity of Apperception}
What makes the synthetic unity of apperception so essential to the process of cognition? Kant canonicalises it as the basis for the possibility of understanding in respect to intuition. His presentation is as follows:

\begin{enumerate}
  \item The supreme principle for the possibility of all intuition in reference to sensibility was, according to the transcendental aesthetic, that everything manifold in intuition is subject to the formal conditions of space and time.
  \item The supreme principle for the possibility of all intuition in reference to understanding is that everything manifold in intuition is subject to the conditions of the original synthetic unity of apperception.

  \autocite[B137]{hackett}
\end{enumerate}

\noindent
On a surface level, it is clear that all possible cogition (i.e. `intuition in reference to understanding') requires the synthetic unity of apperception. However the more interesting realisation is that this also gives all of our intuitions a fundamentally subjective nature -- there will always have to be an `\emph{I think}' particle attached to our cognitions, a self-referential piece that ties these cognitions to the cognitive subject. It is never possible to \emph{cognise} in a manner where the subject is removed from the object. Hence, \emph{all objective cognition is dependent on the cognising subject}.

This is truly a audacious conclusion, and it is important for the traveller to not completely misunderstand it. For we are \emph{not} saying that all cognition is subjective, in the sense of wayward Idealism. Rather, the presence of the subject in cognition is an a priori necessity for objective cognition in the first place:

\begin{quote}
  The reference to this necessary unity is there even if the judgment itself is empirical and hence contingent -- e.g., Bodies are heavy. By this I do not mean that these presentations belong necessarily to one another in the empirical intuition. Rather, I mean that they belong to one another by virtue of the necessary unity of apperception in the synthesis of intuitions

  \autocite[B142]{hackett}
\end{quote}

\noindent
It is through the subject, that different intuitions are brought into unity. Without this synthetic unity of apperception, different terms cannot be brought into commensuration by a common measure, allowing no objective determination, only associative ones. Kant elaborates:

\begin{quote}
  Only through this [reference to the original apperception and it's necessary unity] does this relation [among presentations] become a \emph{judgement}, i.e. a relation that is \emph{valid objectively} and can be distinguished adequately from a relation of the same presentations that would only have a subjective validity -- e.g., a relation according to the laws of association. According to these laws, all I could say is: When I support a body, then I feel a pressure of heaviness. I could not say: It, the body, is heavy -- which amounts to saying that these two presentations are not merely together in perception \ldots\ but are combined in the object, i.e., combined independently of what the subject's state is.

  \autocite[B143]{hackett}
\end{quote}

This transcendental process, in which the presence of the subject (in the form of the synthetic unity of apperception) is contained within the cognitive process, but yet somehow `cancels out' the subjectivity of presentations -- allowing true objective cognition -- is a gymnastic leap of thought of Comănecic\footnote{Nadia Comăneci, a Romanian gymnast who achieved the first perfect score at the Olympics.} proportions. It is important that we understand this synthetic unity of apperception, for it speaks to the deep nature of how cognition works. A simplified demonstration can be given in the following manner:

% Sofia wrote this note:
% dear shen your company and friendship are very dear to me. go and fim uh finish your eesay ;)

\begin{enumerate}
  \item Through the faculty of sensibility, we yield the manifolds of intuition.
  \item Through the faculty of understanding, we yield conception.
  \item We begin the synthetic process of merging sensibility and understanding:
  \begin{enumerate}
    \item The manifolds of intuition, which are subjective to the sensible self, contain the a priori of space, time, and the synthetic unity of apperception.
    \item The a priori categories behind concepts are by necessity, subjective to the judgemental self.
    \item The two subjective elements are combined synthetically through their common denominator.
  \end{enumerate}
  \item The resulting union of intuition and concept yields cognition
\end{enumerate}

\noindent
Thus concludes Kant's treatment synthetic unity of apperception, the only means in which cognition (including objective cognition) can arise. It is the synthesising process of the transcendental elements of sensibility (aesthetics) and understanding (logic), that is only possible "through the necessary reference of the manifold of intuition to the one [self]" \autocite[B140]{hackett}.

\section*{The Kantian Theory of Mind}
We are now at the summit. The rarefied atmosphere of these transcendental heights have melted away all contingency. These transcendental elements, like the rays of an ascendent sun, shine brilliantly upon our visage, unobstructed by the smog and haze of a posteriori pollution. Our journey was a difficult ascent, one which leaves us with a breathless countenance -- weary from the exertions of our guide's transcendental analytic. And speaking of which \ldots\ where \emph{is} our guide? Look up and turn around! -- but alas, he's nowhere to be seen. We're left alone, with these precepts, and only the ghost of a smile as our transcendental Virgil leaves us. There are further peaks, and farther ranges\footnote{The transcendental dialectic in particular, forms the bulk of Kant's remaining treatise -- but it is only an exposition of the consequences of the analytic. We will only reference it's findings.} -- but none matter to us now, other than the view which is ahead. It is now time to contemplate.

% Talk about the following topics:

% The relationship between the synthetic unity of apperception and self-consciousness

% Whether or not the self-perceptive act is necessary for all cognitive agents

% The objective/subjective nature of the empirical sciences

% Whether or not metaphysical knowledge is limited by reason (?)

% Note: the conclusion is located at ./sections/conclusion.tex
