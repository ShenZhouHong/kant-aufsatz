\section*{Introduction}
\emph{How is mathematics possible? How is physics possible? Is metaphysics as a science possible at all?} These are some of the questions which Immanuel Kant's \emph{Critique of Pure Reason} attempts to answer, through a systematic, and entirely novel presentation of a \emph{transcendental} metaphysics. Over the course of eight hundred-odd pages, Kant examines the human faculties of \emph{sensability}, \emph{understanding}, and \emph{reason}, culminating in a complete metaphysical system with the ambition to serve as a complete foundation to all physics and a posteriori knowledge. Just exactly how successful is this Kant's endeavour, which even he himself likens to a `Copernican Revolution' What \emph{exactly} is the `synthetic unity of apperception?' What do the dialectical `illusions' of pure reason hold, for the rest of Philosophy? And are the mathematical and physical sciences truly \emph{empirical}, under the terms of Kant's transcendental foundation? This paper seeks to explore these questions, and more -- in a critical discourse of Kant's Critique of Pure Reason. Thus, I present: A Critique of the Critique of Pure Reason.
